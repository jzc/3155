\documentclass[11pt, letter]{article}
\usepackage[margin=1in]{geometry}
\usepackage{amsmath}
\usepackage{ mathrsfs }
\title{Lab 6 Writeup}
\author{Justin Cai}

\begin{document}
    \maketitle

    \section{Regular Expression Parser: Recursive Descent Parsing}
    \begin{itemize}
        \item [i.] \textit{In your write-up, give a refactored version of the re grammar from Fig-
        ure 1 that eliminates ambiguity in BNF (not EBNF).}
        \begin{align*}
            re &::= union\\
            union &::= intersect \, | \, union \, ``|" \, intersect \\
            intersect &::= concat \, | \, intersect \, ``\&" \, concat \\
            concat &::= not \, | \, concat \, not \\
            not &::= star \, | \, ``\sim" \, not \\
            star &::= atom \, | \, star \, (``*" \, | \, ``+" \, | ``?") \\
            atom &::= c \, | \, ! \, | \, \# \, | \, . \, | \, ``(" \, re \, ``)"
        \end{align*}
        \item [ii.] \textit{Explain briefly why a recursive descent parser following your grammar
        with left recursion would go into an infinite loop.}
        Consider the simpler grammar of expressions:
        \begin{align*}
            e ::= n \mid e + n
        \end{align*}
        where $n$ are numbers. To parse the string ``1+2,'' we would check to see if the string
        matches $n$, which it does not, so then it would check to see if the string matches $e+n$, 
        which involves checking to see the string matches an $e$, then $+$, and $n$. The problem is 
        that the we initally see if the string matches an $e$, this invokes another recursive check to 
        match an $e$, an it will infinitely recurse.
        \item [iii.] \textit{In your write-up, give a refactored version of the re grammar that re-
        places left-associative binary operators with n-ary versions using EBNF.}
        \begin{align*}
            re &::= union\\
            union &::= intersect \, \{ ``|" \, intersect \} \\
            intersect &::= concat \, \{ ``\&" \, concat \}\\
            concat &::= not \, \{not\}\\
            not &::= star \, | \, ``\sim" \, not \\
            star &::= atom \, \{ ``*" \, | \, ``+" \, | ``?" \} \\
            atom &::= c \, | \, ! \, | \, \# \, | \, . \, | \, ``(" \, re \, ``)"
        \end{align*}
        \item [iv.]\textit{In your write-up, give the full refactored grammar in BNF without left
        recursion and new non-terminals like unions for lists of symbols.}
        \begin{align*}
            re &::= union\\
            union &::= intersect \, unions \\
            unions &::= \epsilon \, | \,  ``|" \, intersect \, unions\\
            intersect &::= concat \, intersects\\
            intersects &::= \epsilon \, | \, ``\&" \, concat \, intersects\\
            concat &::= not \, concats \\
            concats &::= \epsilon \, | \, not \, concats \\
            not &::= star \, | \, ``\sim" \, not \\
            star &::= atom \, stars \\
            stars &::= \epsilon \, | \, (`*" \, | \, ``+" \, | \, ``?") \, stars \\
            atom &::= c \, | \, ! \, | \, \# \, | \, . \, | \, ``(" \, re \, ``)"
        \end{align*}
    \end{itemize}

    \section{Regular Expression Literals in JavaScripty}
    \begin{itemize}
        \item [i.] \textit{In your write-up, give typing and small-step operational semantic rules
        for regular expression literals and regular expression tests based on the informal
        specification given above. Clearly and concisely explain how your rules enforce
        the constraints given above and any additional decisions you made.}

        Typing rules:
        \begin{align*}
            \textsc{TypeRe} &: \frac{re \in \mathscr{L}(re)}{\textit{/\^{}re\$/} : \textbf{RegExp}}\\\\
            \textsc{TypeReTest} &: \frac{\Gamma \vdash e_1 : \textbf{RegExp} \qquad \Gamma \vdash e_2 : \textbf{string}}{e_1.test(e_2) : \textbf{bool}}
        \end{align*}
        \textsc{TypeRe} adds typing for a regular expression literal, with the previous judgement that the
        $re$ is a valid regular expression. \textsc{TypeReTest} says that the expression $e_1.test(e_2)$ has 
        a type of bool, given that $e_1$ is a regular expression and $e_2$ is a string.

        Small-step operational semantics:
        \begin{align*}
            \textsc{SearchReTest} &: \frac{e_2 \to e_2'}{\textit{/\^{}re\$/}.test(e_2) \to \textit{/\^{}re\$/}.test(e_2')}\\\\
            \textsc{DoReTest} &: \frac{b' = retest(\textit{/\^{}re\$/}, s)}{\textit{/\^{}re\$/}.test(s) \to b'}
        \end{align*}
        \textsc{SearchCall} will step $e_1$ in $e_1(e_2)$ to $\textit{/\^{}re\$/}.test$ (or \texttt{Call(GetField($e_1$,"test"), List($e_2$))}
        as an AST node). In order for the interpreter to step this, these two rules need to be implemented.
        If $e_2$ is not fully stepped to a string, then \textsc{SearchReTest} will step $e_2$. Once $e_2$
        is stepped to a string, then \textsc{DoReTest} will perform the test by calling the \texttt{retest} 
        function.
    \end{itemize}
    % \section{Scala Basics: Binding and Scope}
    % \begin{itemize}
    %     \item [a.] The use of \texttt{pi} on line 4 is bound at line 3.
    %     The scope in which that use of \texttt{pi} has a definition within that scope, so 
    %     that binding is used. The use of \texttt{pi} on line 7 is bound at line 1. The 
    %     \texttt{area} function does not create a new scope, and the scope in which \texttt{area}
    %     is defined has a definition of \texttt{pi} on line 1, so that one is used.
    %     \item [b.] The use of \texttt{x} on line 3 is bound at line 2. The function parameters of \texttt{f} 
    %     are the innermost definition of \texttt{x}, so that binding is used.
    %     The use of \texttt{x} on line 6 is bound on line 5. \texttt{x} is not bound in that scope, so we use
    %     the binding of \texttt{x} provided by the \texttt{case} statement.
    %     The use of \texttt{x} on line 10 is bound on line 5. This is the same as the use on line 6.
    %     The use of \texttt{x} on line 13 is bound on line 1. This is the outermost scope, so there are
    %     no bindings to shadow and there is a binding provided on line 1, so that one is used.
    % \end{itemize}

    % \section{Scala Basics: Typing}
    % The body of \texttt{g} is well-typed, and returns a \texttt{((Int, Int), Int)}. 
    
    % \texttt{if (x==0) (b, 1) else (b, a+2): ((Int, Int), Int)} because\\
    % \texttt{(b, 1): ((Int, Int), Int)} \\
    % \texttt{(b, a+2): ((Int, Int), Int)} \\
    % \texttt{x: Int}\\
    % \texttt{3: Int}\\
    % \texttt{(x, 3): (Int, Int)} \\
    % \texttt{b: (Int, Int)} \\
    % \texttt{1: Int} \\
    % \texttt{a: Int} \\
    % \texttt{2: Int} \\
    % \texttt{a+2: Int}
\end{document}