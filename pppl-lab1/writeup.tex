\documentclass[11pt, letter]{article}
\usepackage[margin=1in]{geometry}
\title{Lab 1 Writeup}
\author{Justin Cai}

\begin{document}
    \maketitle

    \section{Scala Basics: Binding and Scope}
    \begin{itemize}
        \item [a.] The use of \texttt{pi} on line 4 is bound at line 3.
        The scope in which that use of \texttt{pi} has a definition within that scope, so 
        that binding is used. The use of \texttt{pi} on line 7 is bound at line 1. The 
        \texttt{area} function does not create a new scope, and the scope in which \texttt{area}
        is defined has a definition of \texttt{pi} on line 1, so that one is used.
        \item [b.] The use of \texttt{x} on line 3 is bound at line 2. The function parameters of \texttt{f} 
        are the innermost definition of \texttt{x}, so that binding is used.
        The use of \texttt{x} on line 6 is bound on line 5. \texttt{x} is not bound in that scope, so we use
        the binding of \texttt{x} provided by the \texttt{case} statement.
        The use of \texttt{x} on line 10 is bound on line 5. This is the same as the use on line 6.
        The use of \texttt{x} on line 13 is bound on line 1. This is the outermost scope, so there are
        no bindings to shadow and there is a binding provided on line 1, so that one is used.
    \end{itemize}

    \section{Scala Basics: Typing}
    The body of \texttt{g} is well-typed, and returns a \texttt{((Int, Int), Int)}. 
    
    \texttt{if (x==0) (b, 1) else (b, a+2): ((Int, Int), Int)} because\\
    \texttt{(b, 1): ((Int, Int), Int)} \\
    \texttt{(b, a+2): ((Int, Int), Int)} \\
    \texttt{x: Int}\\
    \texttt{3: Int}\\
    \texttt{(x, 3): (Int, Int)} \\
    \texttt{b: (Int, Int)} \\
    \texttt{1: Int} \\
    \texttt{a: Int} \\
    \texttt{2: Int} \\
    \texttt{a+2: Int}
\end{document}